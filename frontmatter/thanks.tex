%!TEX root = ../dissertation.tex
% the acknowledgments section

\newthought{The past six years at Harvard} have been some of the best in my life. The people I've met have become some of my best friends. The colleagues I've worked with, both inside and outside of the lab, have pushed me to grow and continue to inspire me to think bigger than I do. This dissertation has been influenced by so many people that sitting down to write this acknowledgement seems more overwhelming than the dissertation itself. Perhaps I should start with the single most influential person in my life, the person to whom this text is dedicated. Samantha Cardillo has both supported and encouraged me throughout my time at Harvard and before. We were traveling together when we first found out that I was accepted to grad school and she celebrated, even knowing it meant 6+ years of stress, long hours, and late nights apart. We moved from sunny California to Boston together where she helped me through the hardest times.  Above all others, this thesis would not have been possible without her. 

While my path to graduation was certainly a windy one, I could not have asked for a better advisor and mentor than Philip Kim. Philip has an unbelievable ability to provide support and understanding while still applying enough guidance and direction to keep me moving. I don't think I've ever met someone who cared more deeply for the well-being of their students than Philip. Plus, probably the best meal I've had in years was at one of his now famous BBQs and how many other people get to say they beat their advisor at candle pin bowling.

It would be an understatement to say that this dissertation would not have happened without the help of KC Fong. KC has affected, in some way, nearly every experiment in this thesis. He helped design, guide, and interpret the Johnson noise thermometry experiments which realized the observation of the Dirac fluid and it is safe to say he has taught me everything I know about microwave experiments. More than this though, I consider KC to be one of my closest friends.

I want to thank all the members of the Kim lab for helping me along the way. Jonah, Ke, Gil-Ho, Xiaomeng, Jing, and Frank, these experiments only worked thanks to you sacrificing your time scratching, transferring, fabricating, and brainstorming. Artem, Kemen and Hugo, the effort you guys spent on building and optimizing the RF circuits and cyrostats was invaluable. Austin and Andy, I likely would have gone crazy over the years without you guys there to distract me. I hope you all take it as a compliment that one major reason I had to write this thesis from home was because, ``I have WAY too many friends in lab to write productively"

Over the years, I was lucky enough to collaborate with some amazing people. In particular, I'd like to thank Andy Lucas and Subir Sachdev for taking the time to really explain hydrodynamics in terms that an experimentalist could understand. Thank you to all the folks at the Raytheon BBN, especially Tom, Blake, Colm, and Graham, for going out of their way to patiently teach me how an experiment should be run and Marcus, Mohammed, and Zach for all the great discussions over the years. Thank you to the Yacoby and Capasso labs who were always there to lend tools, cryogenics, and, most importantly, their expertise. And of course, Hannah Belcher, Carolyn Moore, and Bill Walker for doing more than any person should to help me navigate the Harvard bureaucracy.

I can not overstate the great undergraduate education I received from the University of California at Santa Barbara and the Santa Barbara City College. In particular, I owe my deepest gratitude to Mike Young and Jim Allen for showing me what physics was and convincing me to give my education a second chance and to Deborah Fygenson all her guidance over the years. Everyone knows an undergraduate researcher is only as good as their graduate mentor and I was lucky enough to have two great ones. Greg Dyer and Kim Weirich are two of the most patient and helpful people I've known and I'm happy to now call them both friends. As any physics student who has gone through Santa Barbara City College can tell you, at the heart of that program is Don Ion who seemed to teach more with a question than any book could with a paragraph.

I want to thank the amazing friends I've made over the past years who have made my time in Boston great. Bryan Kaye and Daniel Wintz have become two of my best friends and there isn't a doubt in my mind I would not have made it through without them. It's hard to imagine a better group of people than Zach Gault, Will Fitzhugh, Bryan Hassell, Pete Greskoff, and Michael Brady. Evan Walsh, Tony Zhou, Danny Kim, Tommi Hakala, Rui Zhang, and Xi Wang were always there for me. And even though they were mostly on the other side of the country, I want to thank Steve Crawford, Brian Hoffman, Lucy Rangel, Will Snyders, Tommy Foley, Amber Mccreary, Kyle Lutz, Norah Olley, Kevin Dober, Stephanie Hall, Greg Tavangar, Daniel Murawka, and Alex Woolf for reminding me to keep my eye on graduation. 

Finally, I cannot say enough about the love and support of family. Without the continual efforts of my parents, Mark and Sally Crossno, and brother, Jordan Crossno, none of this would have been possible. 