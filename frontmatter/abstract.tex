%!TEX root = ../dissertation.tex
This dissertation presents the methods and experimental results of studies on electronic thermal transport in mesoscopic conductors by means of radio frequency Johnson noise thermometry. In particular, we present the application of these methods to study the electronic thermal conductivity of monolayer graphene over a wide range of temperature, charge density, and magnetic field.

A comprehensive theory of thermal noise in conductors is formulated in a language convenient for high frequency measurements of mesoscopic samples. Employing low noise amplifiers, $2~GHz$ auto- and cross-correlated Johnson noise measurements are performed in the temperature range of $3$ to $300~K$, achieving a sensitivity of $5.5~mK$ ($110~ppm$) in $1~s$ of integration time. Techniques for overcoming the challenges of measuring devices with resistances that dynamically vary over multiple orders of magnitude are presented. Impedance matching circuits compatible of withstanding the harsh measurement environments while remaining stable across the extreme changes in temperature and magnetic field are described. A systematic and robust calibration procedure for converting noise power to electronic temperature is outlined which allows for the inevitable drift of device resistance that often plagues condensed matter experiments. With the ability to measure electronic temperature the thermal conductance between the electronic system and a thermal bath can be measured. 

We quantitative discuss the various cooling mechanisms of the quasi-relativistic electrons in graphene and how they combine to create a complicated thermal network. Moreover, we present experimental techniques to disentangle these mechanisms allowing the study of each cooling pathway independently. Using these methods, the electron-phonon coupling of clean graphene is quantitatively compared to theoretical estimates and found to an order of magnitude larger than that predicted for graphene intrinsic acoustic-phonons and the disorder-assisted supercollision mechanism. We also find that at low temperature and high carrier density, the thermal conductivity of diffusive monolayer graphene closely obeys the Wiedemann-Franz law. 

Near the charge neutrality point at intermediate temperatures, we present evidence that the electronic system in monolayer graphene forms an electron-hole plasma with collective behavior described by hydrodynamics. This charge-neutral plasma of quasi-relativistic fermions, known as a Dirac fluid, exhibits a substantial enhancement of the thermal conductivity, due to decoupling of charge and heat currents. We report an order of magnitude increase in the thermal conductivity and the breakdown of the Wiedemann-Franz law in the thermally populated charge-neutral plasma in graphene. A novel hydrodynamic framework in the presence of charge disorder --- in the form of a static spatially varying chemical potentials --- is presented and compared quantitatively to our experimental results.

Lastly, measurements for the low temperature thermal conduction of graphene under a magnetic field are presented. We report data spanning from zero field, through the semi-classical quantum oscillation regime, and into the quantum Hall regime.